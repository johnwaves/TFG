\chapter{Diseño general y herramientas}


%%%%%%%%%%%%%%%%%%%%%%%%%%%%%%%%%%%%%%%%%%%%%%%%%%%%%%%%%%%%%%%%%%%
\section{Entorno de diseño general}
Tras analizar detenidamente las especificaciones y los requisitos del sistema, se ha decidido desarrollar el proyecto en un entorno web en lugar de crear una aplicación nativa para dispositivos móviles. Se presentan las razones que sustentan esta decisión:

\begin{enumerate} 
	\item \textbf{Accesibilidad multidispositivo y compatibilidad (RFN4.3)}
	Un entorno web permite que el sistema sea accesible desde una variedad de dispositivos, como ordenadores de sobremesa, portátiles, tabletas y una amplia gama de \textit{smartphones}. Esto es especialmente relevante porque tanto el personal sanitario como los pacientes pueden utilizar diferentes dispositivos para acceder al sistema. La compatibilidad con múltiples plataformas facilita el cumplimiento del requisito de usabilidad y ofrece flexibilidad de acceso desde cualquier lugar.
	
	\item \textbf{Facilidad de actualización y mantenimiento (RFN7)}
	
	El desarrollo web centraliza el mantenimiento y las actualizaciones del sistema. Cualquier mejora o corrección se implementa directamente en el servidor, y los cambios se reflejan de inmediato para todos los usuarios sin que se requieran acciones adicionales por su parte. Esto cumple con el requisito de mantenimiento, que exige actualizaciones sin interrupciones en las operaciones críticas.
	
	\item \textbf{Usabilidad para el personal sanitario (RFN4)}
	
	El personal sanitario en las farmacias está acostumbrado a utilizar sistemas basados en navegadores web en ordenadores de sobremesa o tablets. Una interfaz web intuitiva facilita la adopción del sistema y reduce la curva de aprendizaje, alineándose con el requisito de usabilidad que considera el nivel de familiaridad con las tecnologías existentes.
	
	\item \textbf{Seguridad y cumplimiento normativo (RFN1 y RFN6)}
	
	Dado que el sistema manejará información médica y personal sensible, la seguridad es un aspecto fundamental. Un entorno web permite implementar medidas de seguridad robustas en el servidor, como cifrado de datos y autenticación segura, controlando eficazmente el acceso y protegiendo los datos conforme a normativas vigentes como el Reglamento General de Protección de Datos (RGPD). El control centralizado facilita el cumplimiento normativo y la aplicación coherente de políticas de seguridad.
	
	\item \textbf{Disponibilidad y rendimiento (RFN2 y RFN3)}
	
	Un sistema web alojado en servidores confiables puede garantizar una alta disponibilidad y un rendimiento óptimo, aspectos esenciales para las operaciones continuas en las farmacias comunitarias. La infraestructura puede escalarse según sea necesario para manejar altos volúmenes de transacciones y consultas sin afectar el rendimiento, satisfaciendo así los requisitos de disponibilidad y rendimiento.
	
	\item \textbf{Escalabilidad e integración (RFN5)}
	
	El entorno web facilita la escalabilidad del sistema para adaptarse al crecimiento en el número de usuarios y farmacias integradas. Nuevas farmacias y personal pueden añadirse sin necesidad de desplegar o actualizar aplicaciones en múltiples dispositivos, lo que agiliza la expansión y adopción del sistema.
	
	\item \textbf{Gestión centralizada de usuarios (RF6)}
	
	La gestión de usuarios es más eficiente en un entorno web, donde las cuentas y permisos pueden administrarse de forma centralizada. Esto permite mantener un control adecuado sobre el acceso al sistema y garantiza que sólo el personal autorizado tenga acceso a la información sensible.
	
	\item \textbf{Acceso para pacientes y tutores o acompañantes}
	
	Dado que el sistema también será utilizado por pacientes y, en algunos casos, por sus tutores o acompañantes, el acceso web elimina la necesidad de descargar e instalar una aplicación móvil. Esto facilita la participación de los pacientes en el seguimiento de sus tratamientos y mejora la comunicación con el personal sanitario.
	
	\item \textbf{Costos y tiempo de desarrollo}
	
	El desarrollo de una aplicación web suele ser más rápido y económico que crear aplicaciones nativas para múltiples plataformas móviles (iOS, Android). Esto permite optimizar los recursos y cumplir con los plazos del proyecto sin sacrificar la calidad o funcionalidad del sistema.
	
	\item \textbf{Evitar dependencia de plataformas y tiendas de aplicaciones}
	
	Las aplicaciones móviles dependen de las tiendas de aplicaciones para su distribución, lo que puede introducir retrasos y requisitos adicionales para su aprobación. Además, existen políticas y restricciones que pueden afectar la implementación de ciertas funcionalidades. Un entorno web elimina estas dependencias, permitiendo un control total sobre el despliegue y las actualizaciones del sistema.
	
	\item \textbf{Facilidad de integración con sistemas existentes}
	
	Las farmacias pueden contar con sistemas existentes con los que el nuevo sistema debe interactuar. Las aplicaciones web son más flexibles para integrarse con otros sistemas y bases de datos, facilitando la interoperabilidad y cumpliendo con los requisitos de escalabilidad e integración.
	
	\item \textbf{Consistencia en la experiencia de usuario}
	
	Un diseño web responsivo garantiza una experiencia de usuario consistente en todos los dispositivos. Esto es importante para mantener la eficiencia en las operaciones diarias del personal sanitario y en la interacción de los pacientes con el sistema.
	
\end{enumerate}

%%%%%%%%%%%%%%%%%%%%%%%%%%%%%%%%%%%%%%%%%%%%%%%%%%%%%%%%%%%%%%%%%%%
\section{Selección de herramientas y tecnologías para el desarrollo}

En esta sección se presenta la elección de las herramientas y tecnologías que se utilizarán para el desarrollo del sistema, basándose en los requisitos funcionales y no funcionales establecidos. La selección busca cumplir con los objetivos de eficiencia, escalabilidad y seguridad, según las necesidades específicas del proyecto.

\subsection{Base de datos: PostgreSQL}

Para almacenar y gestionar los datos se ha seleccionado \textbf{PostgreSQL}, un sistema de gestión de bases de datos relacionales de código abierto que garantiza el cumplimiento de los estándares SQL, asegurando la integridad y consistencia de los datos. Esta característica es especialmente relevante para manejar información médica (RFN1, RFN2, RFN3, RFN6). Además, su extensibilidad permite definir tipos de datos personalizados y funciones que facilitan la adaptación a las necesidades específicas del sistema, garantizando así su escalabilidad (RFN5). El soporte comunitario de PostgreSQL también asegura que la base de datos reciba actualizaciones y mantenimiento constantes, alineándose con los requisitos de mantenimiento (RFN7).

\subsection{Contenedor de base de datos: Docker Compose}

Para simplificar la configuración, despliegue y mantenimiento de la base de datos \textbf{PostgreSQL} se ha utilizado un contenedor Docker Compose, mediante el cual se permite una fácil replicación del entorno de desarrollo y se asegura que el sistema pueda desplegarse de forma consistente en distintos entornos. Esto contribuye al cumplimiento de los requisitos de disponibilidad (RFN2) y escalabilidad (RFN5), permitiendo que el sistema se adapte a nuevas farmacias y usuarios sin interrupciones en el servicio.

\subsection{ORM: PrismaJS}

Para interactuar con la base de datos de manera eficiente y segura se ha optado por \textbf{PrismaJS} como el ORM (Object-Relational Mapping) del sistema. Esta herramienta proporciona tipado estático y autocompletado, lo cual minimiza errores y mejora la productividad en el desarrollo (RFN7, RFN4). También incluye funcionalidades para gestionar migraciones de esquema de manera segura y eficiente, lo que se ha usado para el mantenimiento y actualización de la estructura de la base de datos. Asimismo, sus consultas optimizadas contribuyen al rendimiento del sistema (RFN3).

\subsection{Entorno de ejecución: Node.js}

\textbf{Node.js} se ha elegido como entorno de ejecución del servidor debido a su modelo basado en eventos, el cual permite manejar un gran número de conexiones simultáneas sin degradar el rendimiento. De esta forma se cumple con los requisitos de rendimiento y escalabilidad (RFN3, RFN5) del sistema. Este entorno permite unificar el stack tecnológico en JavaScript tanto para el lado del cliente como del servidor, mejorando así la productividad y alineándose con los requisitos de mantenimiento (RFN7).

\subsection{Framework Backend: Fastify}

Para el desarrollo del backend se ha optado por \textbf{Fastify}, un framework para Node.js que destaca por resultar una solución ideal para sistemas que requieren tiempos de respuesta mínimos y eficiencia en el procesamiento, satisfaciendo los requisitos de rendimiento del proyecto (RFN3). Una característica que ha resultado primordial ha sido la estructura modular que presenta Fastify gracias a su sistema de plugins, facilitando la expansión del sistema conforme crezca en funcionalidad y permitiendo un fácil mantenimiento (RFN5, RFN7). Además, Fastify incorpora validación de esquemas JSON, lo cual refuerza la seguridad y confiabilidad de las API, alineándose con los requisitos de seguridad (RFN1).

\subsection{Generador de sitios estáticos: Astro}

Para el desarrollo de la interfaz de usuario y generación de contenido estático se ha seleccionado \textbf{Astro} como generador de sitios estáticos. Astro se ha diseñado para crear sitios web rápidos y optimizados, lo cual mejora la experiencia del usuario y cumple con los requisitos de rendimiento (RFN3) al minimizar la carga de JavaScript en el cliente, lo que se conoce como \textit{Server Side Rendering}. Astro permite utilizar componentes de diferentes frameworks, como React, ofreciendo flexibilidad en el desarrollo del frontend. Esta característica permite construir interfaces de usuario modernas y responsivas, alineándose con los requisitos de usabilidad (RFN4).

\subsection{Biblioteca frontend: React}

Para añadir interactividad en la interfaz de usuario se ha elegido \textbf{React}, una biblioteca de JavaScript ampliamente adoptada, la cual ofrece un enfoque basado en componentes a la par que facilita la creación de interfaces reutilizables y mantenibles (RFN7).

\subsection{Framework de diseño: Tailwind CSS}

Para el diseño y estilización de la interfaz se ha optado por \textbf{Tailwind CSS}. Este framework permite crear diseños gracias a su sistema de utilidades predefinidas, característica que simplifican la personalización de estilos.




