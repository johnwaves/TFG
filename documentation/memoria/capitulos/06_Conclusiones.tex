\chapter{Conclusiones}
El desarrollo de este proyecto ha culminado en la creación de una plataforma que integra un backend escalable y eficiente con un frontend interactivo y funcional, satisfaciendo plenamente todos los requisitos establecidos inicialmente. El sistema es capaz de gestionar de forma segura y efectiva información crítica de pacientes y personal sanitario en el contexto de una farmacia comunitaria. 

\section{Despliegue}
Durante el desarrollo del proyecto se tomó la decisión de mantener esta plataforma como exclusiva para el uso en las farmacias inscritas. Se trata de un soporte orientado a facilitar el trabajo del personal sanitario de las farmacias comunitarias, y la peculiaridad que lo caracteriza es que no todo el público, sin previo registro y alta como paciente de una farmacia, podría tener acceso. Esta característica proporciona una grado más de seguridad para proteger la información personal y médica de cada persona.

\section{Dificultades encontradas}
El proyecto ha enfrentado desafíos significativos, especialmente en lo referente a la configuración e integración de las diversas tecnologías utilizadas. La implementación de Docker Compose para gestionar PostgreSQL y pgAdmin, junto con la integración de Prisma y Fastify, requirió un esfuerzo para garantizar que todos los servicios se comunicaran correctamente y de manera segura. Asimismo, la implementación de controladores y rutas en Fastify, así como la validación de usuarios y permisos, presentó complejidades tanto técnicas como lógicas. Esto fue particularmente relevante durante el desarrollo de mecanismos de autenticación y autorización utilizando JSON Web Tokens (JWT) para controlar el acceso a los datos de los usuarios.

En el ámbito del frontend, la configuración de Astro y su integración con React añadieron niveles adicionales de complejidad, especialmente en la estructuración de componentes y la optimización del rendimiento de la interfaz de usuario. \\


\section{Conclusión final}
A nivel personal, la satisfacción con el resultado obtenido es alta. Gracias a todo el aprendizaje obtenido se podrán ampliar y mejorar todas las funcionalidades que ofrece la plataforma y realizar un despliegue real en el futuro.