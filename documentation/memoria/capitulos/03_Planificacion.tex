\chapter{Planificación}

Para que este proyecto se ejecute con éxito, se requiere una planificación meticulosa basada en una metodología específica, dado el amplio espectro de requisitos que deben cumplirse. En las secciones siguientes se detallan tanto la metodología seleccionada como el cronograma de trabajo propuesto.

%%%%%%%%%%%%%%%%%%%%%%%%%%%%%%%%%%%%%%%%%%%%%%%%%%%%%%%%%%%%%%%%%%%
\section{Metodología a utilizar}
La selección de una metodología de desarrollo adecuada es esencial para el éxito de este proyecto. Dado que buscamos flexibilidad, adaptación a cambios en los requisitos y entrega continua de valor, hemos optado por implementar la metodología ágil \textbf{Scrum}.

\subsection{Introducción a SCRUM}

Scrum es un marco de trabajo que promueve el desarrollo iterativo e incremental, centrado en la colaboración, la comunicación constante y la capacidad de adaptarse rápidamente a cambios. Esta metodología es especialmente adecuada para este tipo de proyectos, donde los requisitos pueden evolucionar y se necesita una entrega temprana de funcionalidades.

Se ha adoptado Scrum de manera adaptada para un desarrollo individual. A pesar de ser un proyecto realizado en solitario, se han integrado los principios y prácticas de Scrum, utilizando historias de usuario propias y los requisitos funcionales y no funcionales, y casos de uso descritos anteriormente.

\begin{itemize}
	\item \textbf{Roles}
	\begin{itemize} 
		\item \textbf{Desarrollador}: Se han asumido las responsabilidades del \textit{Product Owner}, del \textit{Scrum Master} y del equipo de pesarrollo. Se ha gestionado el \textit{Product Backlog}, planificado los sprints y llevado a cabo el desarrollo técnico del proyecto.
		\item \textbf{Tutor}: Ha actuado como \textit{stakeholder} y asesor, proporcionando orientación y retroalimentación en las reuniones periódicas, asegurando que el proyecto cumpla con los objetivos académicos y técnicos. 
	\end{itemize}

\newpage

	\item  \textbf{Artefactos}
	\begin{itemize} 
		\item \textbf{Product Backlog}: Lista priorizada y dinámica de todo lo que se requiere en el producto, incluyendo funcionalidades, mejoras y correcciones, basada en las historias de usuario y requisitos definidos.
		\item \textbf{Sprint Backlog}: Conjunto de elementos del \textit{Product Backlog} seleccionados para ser desarrollados en el sprint actual, junto con un plan detallado para su implementación.
		\item \textbf{Incremento}: Suma de todos los elementos completados durante un sprint, representando un avance tangible en el proyecto.
	\end{itemize}
	
	\item \textbf{Eventos}
	\begin{itemize} 
		\item \textbf{Planificación del sprint}: Al inicio de cada sprint, se han definido los objetivos y las tareas a realizar, priorizando según la importancia y la viabilidad.
		\item \textbf{Revisión del sprint}: Al finalizar cada sprint, se han llevado a cabo reuniones con el tutor para presentar los avances, recibir retroalimentación y ajustar el \textit{Product Backlog} si es necesario.
		\item \textbf{Retrospectiva del sprint}: Se ha reflexionado sobre el proceso de trabajo, identificando áreas de mejora para incrementar la eficiencia y la calidad en sprints futuros.
	\end{itemize}
\end{itemize}


\subsection{Adaptación de Scrum al contexto del proyecto}

Dado que el proyecto se basa en desarrollar un \textit{Producto Mínimo Viable} (MVP) y luego expandirlo, Scrum ofrece las herramientas necesarias para cumplir con este objetivo. Las razones principales por las que se ha escogido este tipo de metodología son:

\begin{itemize} 
	\item \textbf{Priorización de funcionalidades críticas}: Al enfocarse en las historias de usuario y casos de uso más relevantes, se asegura que el MVP cumpla con los requisitos esenciales. 
	
	\item \textbf{Entrega temprana y continua}: Los sprints permiten entregar incrementos funcionales en cortos períodos de tiempo, proporcionando valor desde las primeras etapas del proyecto. 
	
	\item \textbf{Retroalimentación constante}: Las reuniones regulares con el tutor permiten ajustar el desarrollo según sus recomendaciones y expectativas. 
	
	\item \textbf{Gestión efectiva de cambios}: Scrum facilita la incorporación de nuevos requisitos o modificaciones sin afectar significativamente el cronograma general. \end{itemize}


\subsection{Planificación con Scrum}
La implementación de Scrum en el proyecto ha seguido un ciclo iterativo compuesto por sprints de duración de aproximadamente dos semanas. Cada sprint ha incluido las siguientes fases:

\begin{enumerate} 
	\item \textbf{Planificación del sprint}: 
		\begin{itemize} 
			\item Selección de elementos del \textit{Product Backlog} para el sprint. \item Definición de objetivos claros y alcanzables. 
			\item Elaboración de un plan de trabajo detallado. \end{itemize}
	
	\item \textbf{Ejecución del sprint}: 
	\begin{itemize} 
		\item Desarrollo de las funcionalidades seleccionadas.
		\item Seguimiento diario del progreso y ajuste del plan según sea necesario.
	\end{itemize} 
	
	\item \textbf{Revisión del sprint}: 
	\begin{itemize} 
		\item Presentación del incremento al tutor.
		\item Recopilación de retroalimentación para futuras mejoras.
	\end{itemize} 
	
	\item \textbf{Retrospectiva del sprint}: 
	\begin{itemize} 
		\item Evaluación del proceso y las herramientas utilizadas.
		\item Identificación de acciones para mejorar en el siguiente sprint.
	\end{itemize} 
	
\end{enumerate}



%%%%%%%%%%%%%%%%%%%%%%%%%%%%%%%%%%%%%%%%%%%%%%%%%%%%%%%%%%%%%%%%%%%
\section{Organización y planificación temporal}
Las especificaciones del sistema para este proyecto, detalladas en el capítulo anterior, se han organizado por hitos o \textit{milestones}, los cuales agrupan uno o varios casos de uso para seguir la estructura de Scrum. A su vez, estos hitos han incluido etapas de estudio, investigación y documentación.

\begin{itemize}
	\item \textbf{Hito 0}: Inicio del proyecto.
	\begin{itemize} 
		\item Investigación sobre el concepto central del proyecto. 
		\item Estudio del estado del arte y definición del dominio del problema. 
		\item Identificación de actores y requisitos. 
		\item Especificación de necesidades: historias de usuario, requisitos funcionales y no funcionales, y casos de uso. 
		\item Planificación del desarrollo. 
	\end{itemize}

	\item  \textbf{Hito 1}: Elección de herramientas de trabajo.
	\begin{itemize}
		\item Estudio acerca de la implementación deseada.
		\item Comparación y elección de herramientas disponibles.
	\end{itemize}

	\item  \textbf{Hito 2}: Desarrollo del backend.
	\begin{itemize}
		\item Diseño de la base de datos.
		\item Diseño de API.
		\item Implementación de procedimientos de autenticación y autorización para API.
	\end{itemize}

	
	\item  \textbf{Hito 3}: Desarrollo inicial del frontend.
	\begin{itemize}
		\item Bocetos de las vistas.
		\item Registro y login para usuarios.
		\item Primera versión de la página de inicio (landing).
	\end{itemize}

	\item \textbf{Hito 4}: Primeras funcionalidades.
	\begin{itemize}
		\item Implementación de CU01, CU02, CU07 y CU06
	\end{itemize}

	\item \textbf{Hito 5}: Funcionalidades para el seguimiento de la adherencia.
	\begin{itemize}
		\item Implementación de CU03 y CU04.
		\item Implementación de CU07.
	\end{itemize}

	\item \textbf{Hito 6}: Documentación y despliegue.
	\begin{itemize}
		\item Documentación del proceso de desarrollo.
		\item Despliegue del sistema.
	\end{itemize}
\end{itemize}


%%%%%%%%%%%%%%%%%%%%%%%%%%%%%%%%%%%%%%%%%%%%%%%%%%%%%%%%%%%%%%%%%%%
\section{Control de versiones}
La herramienta elegida para controlar las versiones y llevar a cabo un desarrollo organizado del proyecto ha sido GitHub. En el repositorio \texttt{TFG} se encuentra el proyecto: \url{https://github.com/johnwaves/TFG}.

Para la consecución de cada hito, se han creado diversas ramas (\textit{branches}) en el repositorio. Cada rama ha estado dedicada al desarrollo de funcionalidades específicas dentro de la implementación de casos de uso concretos. Al finalizar el trabajo en una rama, se ha generado un \textit{Pull Request} hacia la rama principal (\texttt{main}), permitiendo revisar y fusionar los cambios de manera controlada. \\

Los \textit{Pull Requests} han contenido cambios significativos que representan avances importantes en el proyecto. Antes de fusionarlos, se ha realizado una revisión exhaustiva del código para asegurar la calidad, detectar posibles errores y mantener la coherencia con el resto del sistema. \\

Además, se han utilizado los \textit{issues} para registrar y rastrear tareas pendientes, reportar errores y proponer nuevas funcionalidades. Cada \textit{issue} ha estado asociado a etiquetas (\textit{labels}) y se ha asignado a hitos (\textit{milestones}) correspondientes, mejorando la organización y priorización del trabajo.


	
	\begin{figure}[h!]
		\centering
		\begin{ganttchart}[
			y unit chart=0.60cm,
			x unit=0.60cm,
			vgrid,
			hgrid,
			title label font=\bfseries,
			title label anchor/.style={below=-1.6ex},
			bar/.style={fill=orange},
			bar height=.6
			]{1}{14} % Inicio y fin del cronograma (semana 1 a 14)
			
			% Título del cronograma
			\gantttitle{Semanas desde 01/08 hasta 12/11}{14} \\
			\gantttitlelist{1, 2, 3, 4, 5, 6, 7, 8, 9, 10, 11, 12, 13, 14}{1} \\
			
			% Hitos del proyecto con las nuevas duraciones
			\ganttbar[bar/.append style={fill=yellow}]{Hito 0: Inicio del proyecto}{1}{3} \\
			\ganttbar[bar/.append style={fill=orange}]{Hito 1: Elección de herramientas}{4}{4} \\
			\ganttbar[bar/.append style={fill=red}]{Hito 2: Desarrollo del backend}{5}{7} \\
			\ganttbar[bar/.append style={fill=pink}]{Hito 3: Desarrollo inicial del frontend}{8}{10} \\
			\ganttbar[bar/.append style={fill=blue}]{Hito 4: Primeras funcionalidades}{11}{12} \\
			\ganttbar[bar/.append style={fill=cyan}]{Hito 5: Funcionalidades de seguimiento}{13}{13} \\
			\ganttbar[bar/.append style={fill=green}]{Hito 6: Documentación y despliegue}{14}{14} \\
			
		\end{ganttchart}
		\caption{Diagrama de Gantt.}
	\end{figure}
	

