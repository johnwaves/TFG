\chapter{Planificación}

Para que este proyecto se ejecute con éxito, se requiere una planificación meticulosa basada en una metodología específica, dado el amplio espectro de requisitos que deben cumplirse. En las secciones siguientes se detallan tanto la metodología seleccionada como el cronograma de trabajo propuesto.

%%%%%%%%%%%%%%%%%%%%%%%%%%%%%%%%%%%%%%%%%%%%%%%%%%%%%%%%%%%%%%%%%%%
\section{Metodología a utilizar}
La selección de una metodología de desarrollo adecuada es esencial para el éxito de este proyecto. Dado que buscamos flexibilidad, adaptación a cambios en los requisitos y entrega continua de valor, hemos optado por implementar la metodología ágil \textbf{Scrum}.

\subsection{Introducción a SCRUM}

Scrum es un marco de trabajo que promueve el desarrollo iterativo e incremental, centrado en la colaboración, la comunicación constante y la capacidad de adaptarse rápidamente a cambios. Esta metodología es especialmente adecuada para este tipo de proyectos, donde los requisitos pueden evolucionar y se necesita una entrega temprana de funcionalidades.

\subsubsection{Adopción de Scrum en el proyecto}

Para adoptar Scrum de manera efectiva, se establecerá la siguiente estructura:

\paragraph{Roles}

\begin{itemize} 
	\item \textbf{Product Owner}: Responsable de maximizar el valor del producto y gestionar el \textit{Product Backlog}. Representa los intereses de los usuarios y \textit{stakeholders}. 
	\item \textbf{Scrum Master}: Facilita el proceso Scrum, asegurando que el equipo entienda y aplique correctamente la metodología. Elimina impedimentos y promueve la colaboración. 
	\item \textbf{Equipo de Desarrollo}: Grupo multidisciplinario y autoorganizado encargado de entregar incrementos de producto potencialmente utilizables al final de cada sprint. 
\end{itemize}

\paragraph{Artefactos}

\begin{itemize} 
	\item \textbf{Product Backlog}: Lista priorizada y dinámica de todo lo que se requiere en el producto, incluyendo funcionalidades, mejoras y correcciones. 
	\item \textbf{Sprint Backlog}: Conjunto de elementos del \textit{Product Backlog} seleccionados para ser desarrollados en el sprint actual, junto con un plan para entregarlos. 
	\item \textbf{Incremento}: Suma de todos los elementos del \textit{Product Backlog} completados durante un sprint y el valor de los incrementos de todos los sprints anteriores. 
\end{itemize}

\paragraph{Eventos}

\begin{itemize} 
	\item \textbf{Sprint Planning}: Reunión al inicio de cada sprint para definir qué se entregará y cómo se hará. 
	\item \textbf{Daily Scrum}: Reuniones diarias de corta duración donde el equipo sincroniza actividades y planifica las próximas 24 horas. 
	\item \textbf{Sprint Review}: Sesión al final del sprint para inspeccionar el incremento y adaptar el \textit{Product Backlog} si es necesario. 
	\item \textbf{Sprint Retrospective}: Reunión para reflexionar sobre el sprint finalizado y definir acciones de mejora para el siguiente. 
\end{itemize}

\subsection{Adaptación de Scrum al contexto del proyecto}

Dado que el proyecto se basa en desarrollar un \textit{Producto Mínimo Viable} (MVP) y luego expandirlo, Scrum ofrece las herramientas necesarias para cumplir con este objetivo. Las razones principales por las que se ha escogido este tipo de metodología son:

\begin{itemize} 
	\item \textbf{Priorización de funcionalidades Críticas}: Al iniciar con las historias de usuario y casos de uso más relevantes, se asegura que el MVP cumpla con los requisitos esenciales. 
	\item \textbf{Entrega Temprana y Continua}: Los sprints permiten entregar incrementos funcionales en cortos períodos de tiempo, proporcionando valor desde las primeras etapas del proyecto. 
	\item \textbf{Retroalimentación Constante}: La interacción regular con los usuarios y \textit{stakeholders} permite ajustar el desarrollo según sus necesidades y expectativas. 
	\item \textbf{Gestión Efectiva de Cambios}: Scrum facilita la incorporación de nuevos requisitos o modificaciones sin afectar significativamente el cronograma general. 
\end{itemize}


\subsection{Planificación con Scrum}

La implementación de Scrum en el proyecto seguirá un ciclo iterativo compuesto por sprints de duración fija, la cual será de dos semanas. Cada sprint incluirá las siguientes fases:

\begin{enumerate} 
	\item \textbf{Planificación del sprint}: 
		\begin{itemize} 
			\item Selección de elementos del \textit{Product Backlog} para el sprint. \item Definición de objetivos claros y alcanzables. 
			\item Elaboración de un plan de trabajo detallado. 
		\end{itemize} 
	
	\item \textbf{Ejecución del sprint}: 
		\begin{itemize} 
			\item Desarrollo de las funcionalidades seleccionadas. 
			\item Reuniones diarias (\textit{Daily Scrum}) para seguimiento y ajuste del progreso. 
		\end{itemize} 
	
	\item \textbf{Revisión del sprint}: 
		\begin{itemize} 
			\item Presentación del incremento al \textit{Product Owner} y \textit{stakeholders}. 
			\item Recopilación de retroalimentación para futuras mejoras.
		\end{itemize} 
	
	\item \textbf{Retrospectiva del sprint}: 
		\begin{itemize} 
			\item Evaluación interna del equipo sobre el proceso y las herramientas utilizadas. 
			\item Identificación de acciones para mejorar en el siguiente sprint. 
		\end{itemize} 
	
\end{enumerate}



%%%%%%%%%%%%%%%%%%%%%%%%%%%%%%%%%%%%%%%%%%%%%%%%%%%%%%%%%%%%%%%%%%%
\section{Organización y planificación temporal}
Las especificaciones del sistema para este proyecto, detalladas en el capítulo anterior, se organizarán por hitos o \textit{milestones}, los cuales agruparán uno o varios casos de uso para poder seguir la estructura de Scrum. A su vez, estos hitos contendrán también etapas de estudio, investigación y documentación de la información obtenida. 

\begin{itemize}
	\item \textbf{Hito 0}: Inicio del proyecto.
	\begin{itemize}
		\item Investigación sobre el concepto de adherencia terapéutica.
		\item Estudio del estado del arte y definición del dominio del problema.
		\item Identificación de actores.
		\item Especificación de necesidades: historias de usuario, requisitos y casos de uso.
		\item Planificación del desarrollo.
	\end{itemize}

	
	\item  \textbf{Hito 1}: Elección de herramientas de trabajo.
	\begin{itemize}
		\item Estudio acerca de la implementación deseada.
		\item Comparación y elección de herramientas disponibles.
	\end{itemize}

	\item  \textbf{Hito 2}: Desarrollo del backend.
	\begin{itemize}
		\item Diseño de la base de datos.
		\item Diseño de API.
		\item Implementación de procedimientos de autenticación y autorización para API.
	\end{itemize}

	
	\item  \textbf{Hito 3}: Desarrollo inicial del frontend.
	\begin{itemize}
		\item Bocetos de las vistas.
		\item Registro y login para usuarios.
		\item Primera versión de la página de inicio (landing).
	\end{itemize}

	\item \textbf{Hito 4}: Primeras funcionalidades.
	\begin{itemize}
		\item Implementación de CU01, CU02, CU07 y CU06
	\end{itemize}

	\item \textbf{Hito 5}: Funcionalidades para el seguimiento de la adherencia.
	\begin{itemize}
		\item Implementación de CU03 y CU04.
	\end{itemize}

	\item \textbf{Hito 6}: Funcionalidades avanzadas.
	\begin{itemize}
		\item Implementación de CU07.
	\end{itemize}

	\item \textbf{Hito 7}: Documentación y despliegue.
	\begin{itemize}
		\item Documentación del proceso de desarrollo.
		\item Despliegue del sistema.
	\end{itemize}
\end{itemize}


%%%%%%%%%%%%%%%%%%%%%%%%%%%%%%%%%%%%%%%%%%%%%%%%%%%%%%%%%%%%%%%%%%%
\section{Control de versiones}
La herramienta elegida para controlar las versiones y poder llevar a cabo un desarrollo organizado del proyecto será GitHub. Para la consecución de cada hito, se crearán diversos \textit{Pull Requests}, los cuales contendrán cambios significativos y que permitirán avanzar. 

