\chapter{Introducción}

La adherencia terapéutica hace referencia al grado de concordancia entre el comportamiento de un paciente y las instrucciones y/o recomendaciones recibidas por parte de un profesional de salud en cuanto a la toma de medicamentos, la implementación de cambios dietéticos y de estilo de vida. Esta adherencia resulta crítica tanto para asegurar la eficacia de los tratamientos, así como para optimizar los resultados de salud y minimizar los riesgos asociados a la morbilidad y resistencia a los mismos. 

La labor de seguimiento supone un proceso interactivo entre el farmacéutico y el paciente, por ello, el rol de este último será de vital importancia para una correcta gestión del tratamiento en vigor.



 \cite{pages2018metodos}

%%%%%%%%%%%%%%%%%%%%%%%%%%%%%%%%%%%%%%%%%%%%%%%%%%%%%%%%%%%%%%%%
\section{Problema a resolver}



Es por ello por lo que el proyecto tiene como fin mejorar dicha adherencia mediante el desarrollo de una plataforma para facilitar a los farmacéuticos la elaboración, preparación y seguimiento de la medicación para pacientes, así como la recomendación y advertencia sobre la toma de la misma de acuerdo a directrices específicas. 



%%%%%%%%%%%%%%%%%%%%%%%%%%%%%%%%%%%%%%%%%%%%%%%%%%%%%%%%%%%%%%%%
\section{Estado del arte}

%%%%%%%%%%%%%%%%%%%%%%%%%%%%%%%%%%%%%%%%%%%%%%%%%%%%%%%%%%%%%%%%
\section{Objetivos}
