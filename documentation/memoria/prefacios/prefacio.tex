\chapter*{}
%\thispagestyle{empty}
%\cleardoublepage

%\thispagestyle{empty}

\input{portada/portada_2}


\thispagestyle{empty}

\begin{center}
{\large\bfseries PharmAD: Desarrollo de una plataforma destinada al seguimiento y mejora de la adherencia terapéutica de los pacientes de una farmacia}\\
\end{center}
\begin{center}
MIHNEA IOAN VIDICAN\\
\end{center}

%\vspace{0.7cm}
\noindent{\textbf{Palabras clave}: PharmAD, adherencia, terapéutica, farmacias, pacientes, seguimiento, cumplimiento.}\\

\vspace{0.7cm}
\noindent{\textbf{Resumen}}\\

PharmAD aborda el desafío de la adherencia terapéutica en las farmacias comunitarias, entendida como el grado en que los pacientes siguen las recomendaciones de los profesionales de la salud para asegurar la efectividad de los tratamientos médicos. El proyecto se centra en facilitar el seguimiento de tratamientos farmacológicos y no farmacológicos mediante una plataforma web, permitiendo al personal sanitario gestionar de manera eficiente dichos tratamientos, monitorizar la evolución de los pacientes y brindar soporte en la gestión de sus terapias. \

Con un enfoque en mejorar la comunicación entre los profesionales de la salud y los pacientes, esta iniciativa pretende contribuir a una atención sanitaria más efectiva y a mejorar la calidad de vida de los pacientes, optimizando el cumplimiento de los tratamientos y reduciendo el riesgo de complicaciones asociadas a un seguimiento inadecuado.

\cleardoublepage


\thispagestyle{empty}


\begin{center}
{\large\bfseries PharmAD: Development of a Platform for Monitoring and Improving Therapeutic Adherence in Pharmacy Patients}\\
\end{center}
\begin{center}
MIHNEA IOAN VIDICAN\\
\end{center}

%\vspace{0.7cm}
\noindent{\textbf{Keywords}: PharmAD, adherence, therapeutic, pharmacies, patients, follow-up, compliance.} \\

\vspace{0.7cm}
\noindent{\textbf{Abstract}}\\

PharmAD tackles the challenge of therapeutic adherence in community pharmacies, defined as the extent to which patients follow healthcare professionals' recommendations to ensure effective medical treatments. The project focuses on facilitating the monitoring of both pharmacological and non-pharmacological treatments through a web platform, allowing healthcare personnel to efficiently manage these treatments, track patient progress, and provide support in therapy management. \

By emphasizing improved communication between healthcare professionals and patients, this initiative seeks to contribute to more effective healthcare delivery and enhance patients' quality of life by optimizing treatment adherence and reducing the risk of complications associated with inadequate monitoring.

\chapter*{}
\thispagestyle{empty}

\noindent\rule[-1ex]{\textwidth}{2pt}\\[4.5ex]

Yo, \textbf{MIHNEA IOAN VIDICAN}, alumno de la titulación Grado en Ingeniería Informática de la \textbf{Escuela Técnica Superior
	de Ingenierías Informática y de Telecomunicación de la Universidad de Granada}, con NIE X7527819B, autorizo la
ubicación de la siguiente copia de mi Trabajo Fin de Grado en la biblioteca del centro para que pueda ser
consultada por las personas que lo deseen.

\vspace{6cm}

\noindent Fdo: MIHNEA IOAN VIDICAN

\vspace{2cm}

\begin{flushright}
	Granada, a 12 de noviembre de 2024.
\end{flushright}

\chapter*{}
\thispagestyle{empty}

\noindent\rule[-1ex]{\textwidth}{2pt}\\[4.5ex]

D. \textbf{JOSÉ MARÍA GUIRAO MIRAS}, Profesor del Departamento de Lenguajes y Sistemas Informáticos de la Universidad de Granada.

\vspace{0.5cm}

\textbf{Informa:}

\vspace{0.5cm}

Que el presente trabajo, titulado \textit{\textbf{Título del proyecto, Subtítulo del proyecto}},
ha sido realizado bajo su supervisión por \textbf{MIHNEA IOAN VIDICAN}, y autorizamos la defensa de dicho trabajo ante el tribunal
que corresponda.

\vspace{0.5cm}

Y para que conste, expiden y firman el presente informe en Granada, a 12 de noviembre de 2024.

\vspace{1cm}

\textbf{Los directores:}

\vspace{5cm}

\noindent \textbf{D. JOSÉ MARÍA GUIRAO MIRAS}

\chapter*{Agradecimientos}
\thispagestyle{empty}

       \vspace{1cm}

En primer lugar, este proyecto se lo dedico a mi padre, la persona que más ha ansiado verme terminar y que, tristemente, ya no se encuentra entre nosotros. \\

Me gustaría agradecer a toda mi familia y a mi pareja por todo el apoyo incondicional, la ayuda, el amor y cariño recibido durante todos estos años. Ha sido un largo camino que, por fin, finaliza. \\

También quiero agradecer a mis amigos más cercanos por haber estado siempre a pie de cañón.\\



